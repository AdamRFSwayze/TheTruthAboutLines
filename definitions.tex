\chapter*{Useful Definitions}


\begin{defn2}[Integer]\label{Integer}
	A number is an integer if it has no fractional part. An integer can be either positive or negative. We denote the set of integers as $\mathbb{Z}$. \\$-4, -2, 0, 3,17,  9002$ are all examples of integers. 
\end{defn2}

\begin{defn2}[Real Number]\label{Real Number}
A real number is a value that can be represented on a number line. We denote the set of all real numbers as $\mathbb{R}$.
\end{defn2}


\begin{defn2}[Imaginary Constant]\label{Imaginary Constant}
	The imaginary constant $i$ is defined as $i= \sqrt{-1}$.
\end{defn2}

\begin{defn2}[Complex Number]\label{Complex Number}
	A complex number is any number of the form $a+bi$ where $a$ and $b$ are real numbers and $i$ is the imaginary constant. 
\end{defn2}

\begin{defn2}[Solution Set]\label{Solution Set}
	A number or set of numbers that a variable could be to make the equation true.  A solution to an equation with multiple variables has a value for each variable. 
\end{defn2}


\begin{defn2}[Distributive Property]\label{Distributive Property}
	The distributive property tells us that $a(b+c)=ab+ac=(b+c)a$. 
	
	\noindent
	\emph{There is a well detailed example \hyperref[Distributive Example]{here}.}
\end{defn2}





\begin{defn2}[Degree]\label{Degree}
The degree of a polynomial is the value of the largest non-negative exponent. 

\noindent
\textbf{Ex.} \emph{The degree of the polynomial $g(x)=x^3 +x-8$ is 3.}
\end{defn2}


\chapter*{Appendix}

