\section{Substitution}

Although most people aren't accustomed to the idea, there is more than one way to do most things in math.  This is a good thing!  If the only way to find the solution to a system of equations was to graph them and point at it, it would be pretty difficult to solve a lot of these systems.  This is where substitution comes in.  Consider two equations in slope-intercept form.

$$y = 2x + 4$$
$$y = 3x + 2$$

If we were to find a solution to this system of equations, there would only be one $y$-value, and it would apply for both of the equations.  In other words, we know that $y$ will have the same value for each equation.  Let's say we know that the solution has a $y$-value of $8$.  We would then have, 

$$8 = 2x + 4$$
$$8 = 3x + 2$$

Well, if the right side of the first equation is equal to $8$, and the right side of the second equation is equal to $8$, they must be equal to each other!  This is the driving force of substitution.  If two things have the same value, they're equal to each other.

$$2x + 4 = 3x + 2$$

And this is single linear equation like we had in the second unit!  We can solve this.

$$\begin{array}{rcl}
2x + 4 & = & 3x + 2\\
4 & = & x + 2 \\
2 $ = $ x \end{array}$$

So now we know the solution has an $x$-value of $2$.  Nice.  So what about that $y$-value?  Is it really $8$?  We can plug the $x$-value we just found into one of those equations and find out.  We'll choose the first equation.

$$\begin{array}{rcl}
y & = & 2\times2 + 4\\
y & = & 4 + 4\\
y & = & 8 \end{array}$$

Awesome.  Now we think the solution has an $x$-value of $2$, and a $y$-value of $8$.  If this is true, then $(2,8)$ must be a solution to the second equation as well.  We can check that.

$$\begin{array}{rcl}
y & = & 3x + 2 \\
8 & = & 3\times 2 + 2\\
8 & = & 6 + 2 \end{array}$$

Everything checks out.  We've found a solution to this linear system.

\begin{prblm}
Try using the graphing strategy on this linear system.  Compare the results.
\end{prblm}

For this last problem, the solution had nice, clean integer values for the solution.  The equations were also given in slope-intercept form.  This won't always be the case.

\begin{example}
Solve the system of linear equations.

$$\begin{array}{rcl}
4y - 16x & = & 25 \\
2y + x & = & -1 \end{array}$$

To use the substitution strategy the way we did last time, we want to solve each equation for either $y$ or $x$.  We had the equations solved for $y$ last time, and solving for $y$ should be natural to us by now, so we'll do that for now.

For the first equation:

$$\begin{array}{rcl}
4y - 16x & = & 25 \\
4y & = & 16x + 25 \\
y & = & 4x + \frac{25}{4} \end{array}$$

And for the second equation:

$$\begin{array}{rcl}
2y + x & = & -1 \\
2y & = & -x -1 \\
y & = & -\frac{1}{2}x - \frac{1}{2} \end{array}$$

Now we're in a similar position as we were at the beginning of the last problem.  We realize that both $-\frac{1}{2}x - \frac{1}{2}$ and $4x + \frac{25}{4}$ are equal to $y$, and so we set them equal to each other.

$$\begin{array}{rcl}
-\frac{1}{2}x - \frac{1}{2} & = & 4x + \frac{25}{4}\\
-\frac{1}{2} & = & 4\frac{1}{2}x + \frac{25}{4}\\
-\frac{1}{2} - \frac{25}{4} & = & 4\frac{1}{2}x\\
-\frac{2}{4} - \frac{25}{4} & = & \frac{9}{2}x\\
-\frac{27}{4} & = & \frac{9}{2}x\\
-\frac{27}{4} \divide \frac{9}{2}& = & x\\
-\frac{3}{2} & = & x\\
\end{array}$$

So we've figured out that the solution has an $x$-value of $-\frac{3}{2}$.  Now we need to figure out the $y$-value.  Again, we plug our $x$-value into one of the equations, and the $y$-value should come out.  The great part is that, for the whole sequence where we solved for $y$ in the first equation, each of those equations is equivalent.  We can choose the new version of the first equation to find our $y$-value.

$$\begin{array}{rcl}
y & = & 4\times \left(-\frac{3}{2}\right) + \frac{25}{4}\\
y & = & -\frac{12}{2} + \frac{25}{4}\\
y & = & -\frac{24}{4} + \frac{25}{4}\\
y & = & \frac{1}{4}\\
\end{array}$$

This tells us that the point in the first equation where the $x$-value is $-\frac{3}{2}$ has a $y$-value of $\frac{1}{4}$.  Then, the solution is the point, $\left(-\frac{3}{2}, \frac{1}{4}\right)$.
\end{example}

Let's try a tougher one.

\begin{example}
Solve the system of linear equations.
$$2y - 3x = 9$$
$$5y + 2x = -10$$

For the first equation:

$$\begin{array}{rcl}
2y - 3x & = & 9\\
2y & = & 3x + 9\\
y & = & \frac{3}{2}x + \frac{9}{2} \end{array}$$

And for the second equation:

$$\begin{array}{rcl}
5y + 2x & = & -10\\
5y & = & -10 - 2x\\
y & = & -\frac{2}{5}x - 2 \end{array}$$

Now, setting the right side of each equation equal, 

$$\begin{array}{rcl}
\frac{3}{2}x + \frac{9}{2} & = & -\frac{2}{5}x - 2\\
\frac{3}{2}x + \frac{2}{5}x + \frac{9}{2} & = & - 2\\
\frac{3}{2}x + \frac{2}{5}x & = & -\frac{9}{2} - 2\\
\frac{15}{10}x + \frac{4}{10}x & = & -\frac{9}{2} - \frac{4}{2}\\
\frac{19}{10}x & = & -\frac{13}{2}\\
x & = & -\frac{13}{2} \divide \frac{19}{10}\\
x & = & -\frac{65}{19} \\

\end{array}$$
\end{example}