\chapter{Linear Systems}
\section{Intersections as a Concept}

An important idea in math is the idea of a set.  Sets are collections of objects.  We can talk about the set of flowers, the set of things that are red, or whatever other category of thing we can think of.  If something belongs to both sets, we say that it is in the intersection of those sets.

\begin{prblm}
What's something that's in the intersection of the set of flowers and the set of things that are red?
What would the intersection of the solution sets of two linear equations be like?
\end{prblm}

We've briefly mentioned sets before.

Recall: What is the solution set for the line $y=-x$? What about the line $y=2x+1$?


\begin{defn}[Linear System]
A linear system is made up of multiple linear equations. The solution set for a linear system is the intersection of the solution sets of the linear functions. 
\end{defn}

Over the course of the past few sections, we've been reinforcing this idea that plotting a linear equation as a line makes some features easier to see, such as other solutions, and overall trends of those solutions.  Furthermore, finding the equation that goes with a line that has been plotted gives us some extra power, such as the ability to find solutions that are outside of the plot, or express the facts of the situation in words so that we can communicate these ideas more easily.

\begin{problem}
Find some solutions to the equation $y = 3 - x$.
Find some solutions to the equation $y = 2x$.

How would you find the intersection of their solution sets?
\end{problem}

\section{Graphing Linear Systems}

We've mentioned before that a coordinate system can be created with any scale we want.  The $x$ and $y$ axes don't even need to have the same scale.  That idea requires some refinement now: if we put two lines on the same coordinate system, we have to use them same scale for each line.  When we do this, if we're plotting the line from a set of points, we need to make sure we finish plotting the first line before moving on to the second.

%Insert Illustration

It's no coincidence that we call the objects in two sets the ``intersection''.  A single solution in the solution set of an equation is a point on the line that corresponds to that equation.  If we have two lines on a graph, the point where they intersect is a point that's on both of the lines at the same time.  That means that the point is going to be in the solution set of both equations.

\begin{example}
Find the intersection for the following system of linear equations.
$$y - 4x = -4$$
$$2y + x = 10$$

Using this strategy, we want to plot both equations.  To do this, we want them in a form that we can use to plot them, so we'll put them in slope-intercept form.
For the first line:
$\begin{array}{rcl}
y - 4x & = & -4 \\
y & = & 4x - 4 \end{array}$

And for the second line:
$\begin{array}{rcl}
2y + x & = & 10 \\
2y & = & 10 - x \\ 
y & = & 5 - \frac{x}{2} \\
y & = & - \frac{1}{2} x + 5 \end{array}$

Plotting both lines then shows that they have an intersection.
%Insert graph

We can then see that the lines intersect at the point $(2,4)$.  This means that $(2,4)$ is in the solution set to both equations, and is thereby the solution to this system of equations.
\end{example}

This is a useful tool for finding solutions to systems of linear equations as long as being close is good enough.  Unfortunately, some systems of linear equations have solutions that aren't on the grid-lines for a particular scale.

\begin{prblm}
Consider the following system of linear equations.
$$y = -\frac{1}{2}x + 3$$
$$y = 2x$$

What do you think the intersection is?  What could you do to convince someone who didn't agree?
\end{prblm}
