\pagestyle{plain}
\section*{Ebola Lab}

\begin{exercise}
You're working for the WHO treating ebola in Sierra Leone.  Your facility treats children and adults, sending infants to a specialized facility.  One of your most important medications is for strengthening the heart and maintaining blood pressure.  Unfortunately, you only have 180000 units of medicine for the next three days.  Your goal is to fill as many of your 32 beds as you can, while using as much of your medication as possible, to maximize your resources.  You also have the following facts:

\begin{itemize}
\item Adults require 10 units per pound of body weight.
\item Children require 40 units per pound of body weight to account for smaller hearts.
\item The average child in the region weighs 55 pounds.
\item The average adult in the region weighs 140 pounds.
\end{itemize}

What is the largest number of children you will be able to take into your facility under these constraints?

\end{exercise}
\bigskip

\begin{exercise}
Because of your previous success, the administrators over your facility have called you to advise a decision about funding for a larger operation.  When treatments are successful and nearly complete, or when they are particularly unsuccessful, patients are transferred to another facility.  Furthermore, other hospitals will take note of success and send more patients to this one for treatment.

The administrators have picked up two linear trends relating funding to transfers.  They predict 3 daily transfers out of the facility for every \$8000 above \$60000 in its monthly budget.  They also predict a daily arrival for every \$4000 above \$49500.  They tell you that, if people are transferred out, they want the same number of people coming into the facility so that resources are being used optimally.  How much should they budget for the facility, monthly?

\end{exercise}
\bigskip