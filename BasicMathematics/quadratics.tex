\chapter{Quadratics}

While understanding how to work with lines is incredibly useful for modeling many relationships, we know that the real world is a bit more complex.  When we look at the growth of a population over time or at the path that a ball takes through the air, we quickly realize that linear equations can only do so much.  One of the first step forward is to begin dealing with functions and equations that involve higher exponents.  

This chapter is going to be about equations that are the product of two linear terms.  That is, rather than dealing with functions like $f(x) = 3x - 1$ or $f(x) = \frac{1}{4}x + 2$, we'll be dealing with things like these multiplied together, or $f(x) = (3x-1)(\frac{1}{4}x + 2)$.  We'll cover the different forms that these equations can take, and we'll cover the ways we can solve for our inputs.

The simplest quadratic we deal with is the function which simply squares inputs.  If we put in a 3, we get out a 9.  If we put in a 4, we get out a 16.


\begin{presentation}
	\begin{defn}
	Presentation Defn
	\end{defn}
\end{presentation}

\begin{prblm}
	Solve the system of linear equations:
	
	$$y -2x = -3\frac{1}{2}$$
	$$y = -\frac{3}{2}x + 3\frac{1}{2}$$
\end{prblm}

\begin{theorem}
duh theorem
\end{theorem}

\begin{example}
teh example.
\end{example}

$$2y + 3x = -75$$
$$5y + 8x = -55$$


\section*{Forms of Quadratics}

\subsection*{Factored Form}

If a quadratic is expressed as the product of two linear pieces, we say that it is in factored form.

\begin{defn} Factor

One of the linear terms that is multiplied in the factored form of a polynomial.
\end{defn}

\begin{defn} Root

The roots of a polynomial are the inputs for which it gives an output of zero.  This can also be interpreted as its $x$-intercepts.  When a quadratic is in factored form, it is easy to identify its roots, since only one factor needs to be zero for the whole polynomial to be zero.
\end{defn}

\begin{example}

To find the roots of the quadratic $(x-2)(x+4)$, we need to find where it is equal to zero.  It can only be zero if either $(x-2)$ or $(x+4)$ is equal to zero.  Then the roots correspond to the $x$-intercepts of the linear functions made by each output.

To find the first root, we find out where $(x-2) = 0$.  This is a simple linear equation, and we know that it is true if $x = 2$.

To find the second root, we similarly find out where $(x+4) = 0$.  Here, we know that it is true if $x = -4$.

Since either one of these factors giving an output of zero will make the output of the quadratic zero, both of the solutions we found are roots of the quadratic.

\end{example}


It is important to mention that not every quadratic can be expressed in factored form using only real numbers.  Real numbers have an unfortunate name in that they are not any more real than complex numbers, which have an `imaginary' part.  We won't worry too much about complex numbers in this class.  If a quadratic cannot be expressed in factored form in real numbers.

\subsection*{Standard Form}


One of the most important properties that we regularly use in algebra is the Distributive property.

\begin{example}

In this case, the 3 gets multiplied by each term in parentheses.

$3(x-4) = 3x - 12$

In this case, the -6 gets multiplied by each term in the parentheses, but some simplification is required.

$-6(\frac{1}{4}- x) = -6\frac{1}{4} -(-6x) = -\frac{3}{2} + 6x$

\end{example}

In both of these examples, only a constant was being distributed, but the distributive property works no matter what we're multiplying.  Most importantly, it works with groups of parentheses.

\begin{example}

In the following group, the entire group $(2x + 3)$ gets multiplied by each term in the other group.
$$(2x + 3)(5-x) = (2x+3)5 - (2x+3)x$$

From here, we have two more terms, each of which can be distributed.

$$(2x + 3)5 - (2x + 3)x = 10x+15-2x^2+3x = -2x^2 + 13x + 15$$
\end{example}

Once we have this form of quadratic, we call it a quadratic in standard form.  It is typically ordered from highest exponent to lowest exponent.

\subsection*{Vertex Form}

This form of a quadratic is most closely related to the point-slope form of a line.  Just as with linear functions, switching between the various forms of quadratics can give you different pieces of important information.

\begin{defn} Vertex

A vertex is the lowermost or uppermost point of a parabola.
\end{defn}

To understand vertex form, we first elaborate on the simplest quadratic: $x^2$.  Its vertex will be at the point $(0,0)$.  However, by moving this simple function, called a `parent function', we can make quadratics with vertices at other points.

When thinking about every point as $(x, f(x))$, if we want to move the vertex to the left or the right, we change the $x$-value.  If we want to move the vertex up or down, we change the $f(x)$ value.  Thus, any point where the graph has been moved to the right by $h$ and up by $k$ will be $(x + h, f(x) + k)$.  However, we only want our inputs to be $x$'s, so, adjusting all $x$'s appropriately, $(x, f(x-h)+k)$

\begin{example}

We want to find a quadratic with its vertex at the point $(1,2)$.

First, we look at the parent function, $f(x) = x^2$.  If all points on it are expressed $(x, f(x))$, the vertex is normally at $(0,0)$.  If we want the vertex to be at $(1,2)$, we have to move up one, and over by two.  Thus, our new points will be $(x, f(x-h)+k)$, making our new function $f^\prime(x) = (x-1)^2 + 2$.

This is then a quadratic with a vertex at the point $(1,2)$.

\end{example}

\section*{Factoring Quadratics}



\section*{Solving Quadratics}